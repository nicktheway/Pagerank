H μέθοδος Gauss-Seidel είναι μια αναδρομική μέθοδος επίλυσης γραμμικών συστημάτων. Ο αναδρομικός αλγόριθμός της μεθόδου για την επίλυση ενός γραμμικού συστήματος της μορφής:
$$\bm{A}\vec{x}=\vec{b}$$
είναι για το $i$-στο στοιχείο στο $k+1$ βήμα:
$$x_i(k+1) = \frac{1}{A_{ii}}\bigg(b_i-\sum_{j=1}^{i-1}A_{ij}x_j(k+1)-\sum_{j=i+1}^{N}A_{ij}x_j(k)\bigg)$$

Όπως φαίνεται, ο αλγόριθμος αυτός μοιάζει καθαρά σειριακός αφού ο υπολογισμός του κάθε στοιχείου σε κάθε βήμα εξαρτάται από τον υπολογισμό των νέων τιμών των στοιχείων πριν απ' αυτό.
Για περισσότερες λεπτομέρειες σχετικά με τη μέθοδο αυτή και τα κριτήρια σύγκλισής της ανατρέξτε στον \textcite{saad2003iterative}.

Στην περίπτωσή μας, για την εύρεση της pagerank με τη μέθοδο Gauss-Seidel, θα πρέπει πρώτα να το εκφράσουμε σαν σύστημα γραμμικών εξισώσεων. Όπως προαναφέρθηκε το διάνυσμα Pagerank αποτελεί ιδιοδιάνυσμα του πίνακα $\bm{A}$ για την ιδιοτιμή 1. Επομένως, ισχύει ότι:

\begin{gather*}
\bm{A}_{final}\vec{x} = \vec{x}\Rightarrow (\bm{I}-\bm{A}_{final})\vec{x} = 0 \\
\Rightarrow \bigg(\bm{I}-\frac{b}{N}\cdot\bm{1}_{N\times N}-(1-b)\bm{A}_s\bigg)\vec{x} = 0
\end{gather*}

Όμως, το $\bm{1}_{N\times N}\cdot\vec{x}$ είναι πάντα ίσο με $\bm{1}_{N\times1}$ επειδή το άθροισμα των στοιχείων του $\vec{x}$, ως άθροισμα όλων των πιθανοτήτων, εκφράζει την πιθανότητα ο περιπατητής να βρίσκεται σε μια οποιαδήποτε σελίδα. Έτσι, το σύστημά μας παίρνει τη μορφή:

$$(\bm{I}-(1-b)\bm{A}_s)\vec{x} = \frac{b}{N}\bm{1}_{N\times1}$$

που μπορεί να λυθεί αναδρομικά με τη μέθοδο Gauss-Seidel, αρκεί διασφαλιστεί ότι σε κάθε βήμα $\bm{1}_{N\times N}\cdot\vec{x}_k =\bm{1}_{N\times1}$.
