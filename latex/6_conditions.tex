\begin{itemize}[leftmargin=*]
\item Η διαθέσιμη μνήμη του συστήματος περιορίζει το μέγιστο όριο κόμβων/συνδέσμων που μπορούν να εισαχθούν στο πρόβλημα. Αυτό το όριο περιορίζεται και από τη χρήση 32-bit μεταβλητών για τους κόμβους. Δηλαδή, δεν μπορούν να χρησιμοποιηθούν γράφοι με πάνω από περίπου 4.3δις κόμβους.
\item Τα αρχεία με τους συνδέσμους που επιλέγονται ως είσοδος στα προγράμματα υπολογισμού της pagerank, πρέπει να είναι συγκεκριμένης μορφής για να είναι συμβατά.

Στα παραδοτέα, συμπεριλαμβάνεται ειδικό πρόγραμμα που μετατρέπει αρχεία κειμένου σαν αυτά των \textcite{snapnets} σε συμβατά αρχεία για τα προγράμματα υπολογισμού της pagerank. Το πρόγραμμα αυτό, διαβάζει στην έναρξη των αρχείων τον συνολικό αριθμό κόμβων και απορρίπτει κάθε κόμβο με id μεγαλύτερο αυτού του αριθμού.

\item Η παραλληλοποίηση των iteration της μεθόδου gauss-seidel γίνεται μόνο στις ομάδες που περιέχουν περισσότερους από 50 κόμβους, καθώς το overhead της παραλληλοποίησης κάνει την παράλληλη εκτέλεση του αλγορίθμου μη αποδοτική για μικρότερες ομάδες. Αυτός ο αριθμός, μπορεί εύκολα να αλλάξει ώστε να ταιριάζει στο εκάστοτε σύστημα.

\item Σε ορισμένα web-graphs, στα οποία ο αριθμός των ομάδων που προκύπτει με τον greedy αλγόριθμό είναι αρκετά μεγάλος, ο σειριακός αλγόριθμος θα είναι πιθανότατα ταχύτερος. Χρήση διαφορετικής, αλληλουχίας των κόμβων θα δώσει διαφορετικά αποτελέσματα τόσο στη σύγκλιση όσο και στον αριθμό των διαφορετικών ομάδων.

\item Δεν επιχειρήθηκε αναζήτηση ακολουθίας κόμβων που θα οδηγούσε σε ταχύτερη σύγκλιση ή καλύτερη διαχωρισιμότητα για την παράλληλη υλοποίηση. Όλοι οι κόμβοι διαβάζονται όπως βρίσκονται στο αρχείο και η παραλληλοποίηση γίνεται με τρόπο που δεν επηρεάζει τη σύγκλιση.

\item Τόσο η σειριακή όσο και η παράλληλη υλοποίηση, εκτός των αποτελεσμάτων, δημιουργούν σε κάθε τους εκτέλεση κι ένα αρχείο με δεδομένα χρόνων ή κι άλλα χρήσιμα δεδομένα. Τα διαγράμματα που παρουσιάζονται στην επόμενη ενότητα συγκρίνουν τους καθαρούς χρόνους εκτέλεσης του αλγορίθμου Pagerank των δυο υλοποιήσεων, δηλαδή δεν αφορούν τον απόλυτο χρόνο εκτέλεσης του κάθε προγράμματος.
\end{itemize}


