%\title{Two column article template by M.A.}

\documentclass[11pt, twocolumn]{article}

\usepackage{authblk}

\makeatletter
\renewcommand{\maketitle}{\bgroup\setlength{\parindent}{0pt}
\begin{flushleft}
  \textbf{\@title}

  \@author
\end{flushleft}\egroup
}
\makeatother

\title{\Huge Αλγόριθμος Pagerank με παράλληλη υλοποίηση της μεθόδου Gauss-Seidel \vspace{0.5cm}}
\author{\Large Νικόλαος Κατωμέρης \hspace{2cm} ΑΕΜ: 8551 \hspace{2cm} ngkatomer@auth.gr\normalsize \\ Παράλληλα και διανεμημένα συστήματα. \\Τμήμα ηλεκτρολόγων μηχανικών και μηχανικών υπολογιστών, ΑΠΘ}


\date{\today}

\usepackage[showframe=false, margin=.75in]{geometry}

\usepackage{polyglossia}
\newfontfamily\greekfontsf[Script=Greek]{texgyreheros-regular.otf}
\newfontfamily\greekfonttt[Script=Greek]{Latin Modern Mono}
\setdefaultlanguage{greek}
\setotherlanguage{english}
\setmainfont{Noto Serif}

\usepackage{amsmath}
\usepackage{bm}
\usepackage{abstract}
\renewcommand{\abstractname}{}    % clear the title
\renewcommand{\absnamepos}{empty} % originally center

\usepackage{lipsum}
\usepackage{enumitem}
\usepackage{afterpage}

\renewenvironment{abstract}
 {\small
  \begin{center}
  \bfseries \abstractname\vspace{-.5em}\vspace{0pt}
  \end{center}
  \list{}{%
    \setlength{\leftmargin}{10mm}% <---------- Change margin here
    \setlength{\rightmargin}{\leftmargin}%
  }%
  \item\relax}
 {\endlist}

\usepackage[utf8]{inputenc}
%\usepackage{graphicx}
\usepackage{wrapfig}
\usepackage[lofdepth,lotdepth]{subfig}
\usepackage{booktabs}

\usepackage{titlesec}
\titlespacing\section{0pt}{10pt plus 4pt minus 2pt}{0pt plus 2pt minus 2pt}

\usepackage{rotating}

\usepackage{caption}
%\usepackage{subcaption}

%\usepackage[ngerman]{babel}

\usepackage{csquotes}

\usepackage{todonotes}

\usepackage{stfloats}

%\usepackage[T1]{fontenc}
\usepackage{textcomp}
\usepackage{times}

\usepackage{framed} %um Boxen zu machen

\usepackage[backend=biber, style=ieee]{biblatex}
\addbibresource{references.bib}

%\renewbibmacro*{volume+number+eid}{%
%\printfield{volume}%
%  \setunit*{\adddot}% DELETED
%\setunit*{\addnbspace}% NEW (optional); there's also \addnbthinspace
%\printfield{number}%
%\setunit{\addcomma\space}%
%\printfield{eid}}
%\DeclareFieldFormat[article]{number}{\mkbibparens{#1}}


\usepackage{setspace}
\onehalfspacing

\setlength{\columnsep}{0.8cm}

\setlength{\parskip}{0em}

\usepackage{color}
\definecolor{black}{gray}{0} % 10% gray

\usepackage[colorlinks=true,linkcolor=black,citecolor=black]{hyperref}

\usepackage{tabularx}
\newcolumntype{b}{X}
\newcolumntype{s}{>{\hsize=.5\hsize}X}

\usepackage{ntheorem}
\newtheorem*{TRQ}{Research Question}

\newtheorem{Hyp}{Hypothesis} 


\begin{document}

\twocolumn[
\begin{@twocolumnfalse}
\maketitle
\begin{abstract}
%\textit{\lipsum[2]
%\bigskip}
Στην παρούσα αναφορά περιγράφεται η διαδικασία υλοποίησης και παραλληλοποίησης του αλγορίθμου Gauss-Seidel για τον υπολογισμό του διανύσματος Pagerank γράφων του διαδικτύου. Επίσης, παρουσιάζονται αποτελέσματα σε ενδεικτικούς γράφους καθώς και αποδείξεις ορθότητας του υλοποιημένου παράλληλου προγράμματος.
\end{abstract}
\end{@twocolumnfalse}
]

\section*{Pagerank}
Η βασική ιδέα του αλγορίθμου Pagerank είναι η αξιολόγηση της κάθε σελίδας του διαδικτύου με βάση τον αριθμό των συνδέσμων προς αυτήν και την ποιότητα αυτών, η οποία καθορίζεται από την αξιολόγηση των σελίδων που τους περιέχουν.

Το αποτέλεσμα της παραπάνω αξιολόγησης ταυτίζεται με το ποσοστό του χρόνου στον οποίο ένας τυχαίος περιπατητής θα βρισκόταν στην εκάστοτε σελίδα, αν ξεκινώντας από κάποια τυχαία σελίδα, ακολουθούσε τυχαία κάποιον από τους συνδέσμους της, μετά το ίδιο για την επόμενη κι ούτω καθ' εξής.

Η παραπάνω διαδικασία μπορεί να προσομοιωθεί με μία αλυσίδα Markov της οποίας ο πίνακας μεταβάσεων αποτελείται από στοιχεία που δείχνουν την πιθανότητα μετάβασης από μία σελίδα σε μία άλλη.

Έστω $\bm{A}$, ο $N\times N$  πίνακας μεταβάσεων της αλυσίδας για $N$ σελίδες του διαδικτύου. Αν  $ j\rightarrow i$ σημαίνει ότι η σελίδα $j$ περιέχει σύνδεσμο προς τη σελίδα $i$, ο $\bm{A}$ ορίζεται αρχικά ως εξής:

\[
  \bm{A}(i, j) = 
  \begin{cases}
    1, & \text{αν } j\rightarrow i \\
    0, & \text{αλλιώς }
  \end{cases}
\]

Επειδή όμως ο παραπάνω πίνακας θέλουμε να εκφράζει την πιθανότητα ο περιπατητής να πάει από τη σελίδα $j$ στη σελίδα $i$ σε ένα βήμα, είναι απαραίτητο το άθροισμα των στοιχείων κάθε στήλης του πίνακα να ισούται με $1$.

Οπότε, αφού λάβουμε τον πίνακα με τους συνδέσμους των σελίδων, διαιρούμε όλα του τα στοιχεία με το άθροισμα των στοιχείων της εκάστοτε στήλης $j$, έστω $L_j$, και παίρνουμε τον «στοχαστικοποιημένο» πίνακα $\bm{A}_s$:

\[
  \bm{A}_s(i, j) = 
  \begin{cases}
    \frac{1}{L_j}, & \text{αν } j\rightarrow i \\
    0, & \text{αλλιώς }
  \end{cases}
\]

Επιπλέον, καθώς ο περιπατητής ξεκινά από μία σελίδα, είναι πολύ πιθανό, ακολουθώντας συνεχώς συνδέσμους να είναι αδύνατο να μεταβεί σε όλες τις σελίδες του γράφου. Το γεγονός αυτό, συν το γεγονός ότι ένας χρήστης του διαδικτύου δεν ακολουθεί μόνο συνδέσμους, αλλά «τηλεμεταφέρεται» κιόλας σε άλλες σελίδες, καθιστά τον παραπάνω πίνακα ανεπαρκή για την προσομοίωση της τυχαίας διαδρομής ενός χρήστη του διαδικτύου.

Έτσι, εισάγεται στον αλγόριθμο μια σταθερά «τηλεμεταφοράς» που δηλώνει την πιθανότητα ο περιπατητής να μεταβεί σε κάποια τυχαία σελίδα του διαδικτύου χωρίς να υπάρχει κάποιος σύνδεσμος προς αυτήν στη σελίδα που βρίσκεται.

Η σταθερά αυτή, έστω $c$, μπορεί να επιλεγεί ως όρισμα στο πρόγραμμα υπολογισμού του διανύσματος pagerank που υλοποιήθηκε. Στην περίπτωση που δεν δοθεί, θα έχει την τιμή 0.15 που έχει δειχθεί ότι αποτελεί καλή επιλογή \parencite{brin1998anatomy}.

Με την εισαγωγή της παραπάνω σταθεράς, ο πίνακας μεταβάσεων πλέον θα είναι:
\[
  \bm{A}_{final} = \frac{c}{N}\cdot\bm{1}_{N\times N}+(1-c)\bm{A}_n
\]

Στην παραπάνω σχέση, η σταθερά $c$ διαιρείται με τον αριθμό των γραμμών $N$ έτσι ώστε οι στήλες και του τελικού πίνακα $\bm{A}_{final}$ να έχουν στοιχεία με άθροισμα $1$.

Με τη χρήση του παραπάνω πίνακα είναι δυνατό να υπολογιστεί ένα διάνυσμα $N\times1$ πολλαπλασιάζοντάς τον πίνακα με ένα αρχικό διάνυσμα -πχ ένα ομοιόμορφο- αρκετές φορές, έως ότου το αποτέλεσμα συγκλίνει. Το διάνυσμα αυτό αποτελεί ιδιοδιάνυσμα του πίνακα $\bm{A}$ για την ιδιοτιμή 1 και αποτελεί το ζητούμενο διάνυσμα με την αξιολόγηση της κάθε σελίδας (κόμβου).

Τα παραπάνω δεν θα ισχύουν αν ο πίνακας $\bm{A}_s$ περιέχει σελίδες χωρίς καμία σύνδεση. Σε αυτή την περίπτωση, επιθυμούμε ο περιπατητής να τηλεμεταφερθεί σε μία τυχαία σελίδα. Οπότε, προτού υπολογιστεί ο $\bm{A}_{final}$, είναι απαραίτητο οι μηδενικές στήλες του $\bm{A}_s$ να αντικατασταθούν με νέες στήλες που αποτελούνται από στοιχεία $\frac{1}{N}$. Έστω $\bm{A}_n$ ο πίνακας που δημιουργείται.

Με όλα τα παραπάνω δεδομένα, αποδεικνύεται ότι ο τελικός πίνακας θα έχει μοναδική ιδιοτιμή ένα και το πρόβλημα πλέον έγκειται στην εύρεση του ιδιοδιανύσματός της.

Για την εύρεση του ιδιοδιανύσματος, όπως ζητείται, χρησιμοποιείται η μέθοδος Gauss-Seidel.



\section*{Gauss Seidel}
H μέθοδος Gauss-Seidel είναι μια αναδρομική μέθοδος επίλυσης συστημάτων.
...


\section*{Δεδομένα στη μνήμη}
Το πρόβλημα με την \fref{eq:1} είναι ότι ο πίνακας $(\bm{I}-(1-c)\bm{A}_n)$ δεν είναι αραιός καθώς ο $\bm{A}_n$ , όπως αναφέρθηκε νωρίτερα, έχει γεμάτες τις στήλες με τις σελίδες που στον $\bm{A}_s$ δεν είχαν καμία σύνδεση. Επίσης, το σύνολο αυτών των σελίδων στο διαδίκτυο είναι πολύ μεγάλο\parencite{eiron2004ranking}, με αποτέλεσμα ο $\bm{A}_n$ να απαιτεί πολύ περισσότερο χώρο στη μνήμη από τον $\bm{A}_s$.

Έχουν προταθεί διάφορες λύσεις για την αποφυγή αυτού του φαινομένου, όπως για παράδειγμα να προστεθεί σ' όλες αυτές τις σελίδες σύνδεσμος προς τον εαυτό τους. Μια άλλη πρόταση, η οποία και προτιμήθηκε, χρησιμοποιεί μόνο τον πίνακα $\bm{A}_s$ ο οποίος είναι αραιός και καταλήγει στην ίδια ακριβώς λύση με την \fref{eq:1}. Όπως αποδεικνύεται \parencite{del2005fast}, η λύση της \fref{eq:1} ισοδυναμεί με $\frac{\vec{y}}{\Vert\vec{y}\Vert}_1$, όπου $\vec{y}$ η λύση της:
\begin{equation}\label{eq:2}
(\bm{I}-(1-c)\bm{A}_s)\vec{y} = \frac{1}{N}\cdot\bm{1}_{N\times1}
\end{equation}

H οποία απαιτεί την αποθήκευση μόνο του πίνακα $\bm{A}_s$. Δεδομένου ότι ο πίνακας αυτός είναι πολύ αραιός, αλλά και λόγω του γεγονότος ότι η μέθοδος gauss-seidel έχει, σε κάθε της βήμα και για κάθε κόμβο, πολλαπλασιασμούς στοιχείων μίας γραμμής του $\bm{A}_s$, επιλέχθηκε η χρήση της δομής «Compressed Row Storage (CRS)». 

Η δομή αυτή επιτρέπει την αποθήκευση πινάκων σε χώρο στη μνήμη ανάλογο με τα μη μηδενικά στοιχεία αυτών. Επιπλέον, διατρέχοντας τον πίνακα ανά γραμμή, όλα τα στοιχεία της γραμμής βρίσκονται συνεχόμενα στη μνήμη καθιστώντας τη δομή αυτή φιλική προς την μνήμη cache των επεξεργαστών.

\section*{Σειριακός αλγόριθμος}
Με βάση τα παραπάνω, υλοποιήθηκε το σειριακό πρόγραμμα υπολογισμού της pagerank. Συνοψίζεται ως εξής:
\begin{enumerate}

\item Επιλογή γράφου δεδομένων εισόδου (σύνδεσμοι) και προαιρετικά συγκεκριμένου αριθμού επαναλήψεων ή σταθερών τηλεμεταφοράς ($c$) και σύγκλισης ($E$). Οι προεπιλεγμένες τιμές είναι: \\ $c_{def} = 0.15, E_{def} = 1e-12$

\item Φόρτωση στη μνήμη των δεδομένων εισόδου σε δομή CRS. Tα δεδομένα αυτά αποτελούν τον αρχικό πίνακα $\bm{A}$.

\item Μετατροπή του $\bm{A}$ στον $\bm{A}_s$ και πολλαπλασιασμός του με την σταθερά $-(1-c)$. Έστω $a_{ij}$ τα στοιχεία του πίνακα που προκύπτει.

\item Δημιουργία των $\vec{b},\vec{x}_0=\frac{1}{N}\bm{1}_{N\times1}$

\item Εκτέλεση αναδρομικού αλγορίθμου της μεθόδου Gauss-Seidel έως ότου $\Vert\vec{x}_{k+1}-\vec{x}_k\Vert^2 < E$ ή $k > k_{max}$, όπου έχει προκαθοριστεί $k_{max} = 150$ για ασφάλεια. Επίσης, αν έχει δοθεί συγκεκριμένος αριθμός επαναλήψεων, θα χρησιμοποιηθεί αυτός.

Σε κάθε βήμα του αλγορίθμου:
\begin{itemize}[leftmargin=*]
\item Yπολογίζεται το $x_{k+1}$:

\hfil$\displaystyle x_i(k+1) = \frac{1}{1-a_{ii}}\bigg(b_i-\sum_{j=1}^{i-1}a_{ij}x_j(k+1)-\sum_{j=i+1}^{N}a_{ij}x_j(k)\bigg)$\hfil

\item Υπολογίζεται το $\Vert\vec{x}_{k+1}-\vec{x}_k\Vert^2$ και αυξάνεται το $k$ κατά 1.

\end{itemize}
\item Διαίρεση των στοιχείων του $\vec{x}$ με το άθροισμά τους, ώστε να προκύψει το $\frac{\vec{x}}{\Vert\vec{x}\Vert}_1$ που είναι το διάνυσμα με την pagerank της κάθε σελίδας.
\end{enumerate}

\section*{Παραλληλοποίηση}
Η μέθοδος gauss-seidel γενικώς είναι μια σειριακή μέθοδος. Σε περιπτώσεις όμως που ο πίνακας $\bm{A}$ είναι αραιός, όπως και στην προκειμένη, είναι δυνατός ο παράλληλος υπολογισμός των $x_i$ που δεν αλληλοεξαρτώνται και εξαρτώνται μόνο από στοιχεία $x_j\text{με } j < i$, των οποίων η τιμή έχει ήδη ενημερωθεί, ή στοιχεία που δεν χρειάζεται να έχουν ενημερωθεί ($j > i$).

Για την εύρεση των ομάδων αυτών με τα μη αλληλοεξαρτώμενα στοιχεία του $x$ υλοποιήθηκε ο εξής άπληστος αλγόριθμος:

\begin{itemize}
\item Διατρέχουμε κατά αύξουσα σειρά τα στοιχεία (κόμβους) του γράφου.
\item Τον 1ο κόμβο τον εισάγουμε στην 1η ομάδα.
\item Tον 2ο κόμβο τον εισάγουμε κι αυτόν στην 1η ομάδα εκτός αν υπάρχει σύνδεσμος μεταξύ αυτού και του 1ου. Σε εκείνη την περίπτωση τον τοποθετούμε στην 2η ομάδα.
\item Συνεχίζουμε με τους υπόλοιπους κόμβους, τοποθετώντας τους κάθε φορά στην Λ+1 ομάδα όπου Λ η μεγαλύτερη ομάδα των κόμβων που συνδέονται μ' αυτούς.
\end{itemize}

Αφού, γίνει ο ο «χρωματισμός» των κόμβων σε ομάδες προχωράμε σε ανακατάταξη των κόμβων έτσι ώστε τα στοιχεία της κάθε ομάδας να βρίσκονται συνεχόμενα στη μνήμη και με αύξουσα σειρά ομάδας.

Ύστερα, η εύρεση του διανύσματος pagerank είναι όμοια με τον σειριακό αλγόριθμο με τη διαφορά ότι εδώ υπολογίζουμε παράλληλα τις τιμές $\vec{x}_i(k+1)$ σε κάθε βήμα για τα $i$ που βρίσκονται στην ίδια ομάδα. Επίσης, παράλληλα γίνονται και όλες οι πράξεις πινάκων που απαιτούνται για τον έλεγχο σύγκλισης.

Τέλος, οι κόμβοι επιστρέφουν στην αρχική τους αλληλουχία ώστε να γίνει εύκολα σύγκριση του αποτελέσματος με τον σειριακό αλγόριθμο.


\section*{Συνθήκες και περιορισμοί}
\begin{itemize}[leftmargin=*]
\item Και οι δύο υλοποιήσεις έχουν μέγιστο όριο κόμβων και συνδέσμων με βάση τη διαθέσιμη μνήμη του συστήματος στο οποίο τρέχουν.
\item Τα αρχεία συνδέσμων που δέχονται πρέπει να είναι συγκεκριμένης μορφής. Δηλαδή, θα πρέπει να είναι δυαδικά αρχεία που να περιέχουν με σειρά:
\begin{enumerate}
\item Τον αριθμό κόμβων που περιέχουν.
\item Τον αριθμό συνδέσμων μεταξύ κόμβων που περιέχουν.
\item Όλους τους συνδέσμους με μορφή δύο συνεχόμενων αριθμών που αναπαριστούν τους δύο κόμβους του κάθε συνδέσμου.
\end{enumerate}

Όλοι οι παραπάνω αριθμοί θα πρέπει να έχουν τη μορφή «4-byte uint» για να διαβαστούν σωστά από το αρχείο.

Στα παραδοτέα, συμπεριλαμβάνεται ειδικό πρόγραμμα που μετατρέπει αρχεία κειμένου με κατευθυνόμενους συνδέσμους σε αρχεία της παραπάνω μορφής.

\item Η παραλληλοποίηση των iteration της μεθόδου gauss-seidel γίνεται μόνο στις ομάδες που περιέχουν περισσότερους από 50 κόμβους, καθώς το overhead της παραλληλοποίησης κάνει την παράλληλη εκτέλεση του αλγορίθμου μη αποδοτική για μικρότερες ομάδες.

\item Σε ορισμένα web-graphs, στα οποία ο αριθμός των ομάδων που προκύπτει με τον greedy αλγόριθμό είναι αρκετά μεγάλος, ο σειριακός αλγόριθμος θα είναι πιθανότατα ταχύτερος. Ένας διαφορετικός αλγόριθμος edge-coloring θα δώσει διαφορετικά αποτελέσματα.

\item Τα προγράμματα έχουν υλοποιηθεί για σταθερό web graph. Προσθήκη νέων σελίδων ή αλλαγές σε συνδέσμους, κατά τη διάρκεια του υπολογισμού της pagerank, δεν υποστηρίζονται.

\item Το παράλληλο πρόγραμμα, όπως δίνεται, ενδεχομένως να χρειαστεί περισσότερο χρόνο να τρέξει από το σειριακό καθώς -κάθε φορά που τρέχει- κάνει edge coloring και δύο ανακατατάξεις στον πίνακα και στο τελικό διάνυσμα της pagerank. Αυτοί οι έξτρα χρόνοι δεν είναι απαραίτητοι καθώς οι κόμβοι θα μπορούσαν να είναι εξ' αρχής έτσι στο αρχείο. Παρόλα αυτά μετρούνται χωριστά οι χρόνοι όλων των διαδικασιών και των δύο υλοποιήσεων. Έτσι, τα διαγράμματα που παρουσιάζονται στην επόμενη ενότητα συγκρίνουν τους καθαρούς χρόνους εκτέλεσης του αλγορίθμου Pagerank των δυο υλοποιήσεων.
\end{itemize}




\section*{Ενδεικτικά αποτελέσματα}
Έγινε εκτέλεση των δύο υλοποιήσεων στα web graphs των \textcite{snapnets}.
Εδώ παρουσιάζονται διαγράμματα με διάφορα χαρακτηριστικά των εκτελέσεων για
web graph με 875.713 κόμβους με 4.563.235 συνδέσμους στα ακόλουθα συστήματα:
\begin{enumerate}

\item 4 πύρηνο σύστημα
\begin{itemize}
\item Intel(R) Core(TM) i5-4670 CPU @ 3.40GHz.
\item 8GB DDR3 RAM
\item gcc version 7.3.0
\end{itemize}

\item 8 πύρηνο σύστημα (diades)
\begin{itemize}
\item Intel(R) Xeon(R) CPU E5420  @ 2.50GHz.
\item 8GB DDR3 RAM
\item gcc version 5.4.0
\end{itemize}
\end{enumerate}


\begin{center}
\begin{figure}
\includegraphics[width=\linewidth]{plots/it_time.png}
\caption{Χρονική εξέλιξη των δύο υλοποιήσεων. 4-πύρηνο σύστημα.}
\end{figure}

\begin{figure}
\includegraphics[width=\linewidth]{plots/speed_up.png}
\caption{Σχέση σταθεράς σύγκλισης, επιτάχυνσης και αριθμού επαναλήψεων. 4-πύρηνο σύστημα.}
\label{fig:relat}
\end{figure}

\begin{figure}
\includegraphics[width=\linewidth]{plots/pagerank.png}
\caption{Το αποτέλεσμα για κάθε κόμβο.}
\end{figure}

\begin{figure}
\includegraphics[width=\linewidth]{plots/it_times_diades.png}
\caption{Χρονική εξέλιξη των δύο υλοποιήσεων. 8-πύρηνο σύστημα.}
\end{figure}

\begin{figure}
\includegraphics[width=\linewidth]{plots/speed_up_diades.png}
\caption{Σχέση σταθεράς σύγκλισης, επιτάχυνσης και αριθμού επαναλήψεων. 8-πύρηνο σύστημα.}
\label{fig:relatd}
\end{figure}

\begin{figure}
\includegraphics[width=\linewidth]{plots/diff.png}
\caption{Διαφορά των δύο υλοποιήσεων στο τελικό διάνυσμα.}
\label{fig:dif}
\end{figure}

\end{center}


Αποτελέσματα και για τα υπόλοιπα web-graphs συμπεριλαμβάνονται στα παραδοτέα. Γενικώς, παρατηρήθηκε ικανοποιητική επιτάχυνση με ταχύτητα τουλάχιστον 2 φορές μεγαλύτερη για την παράλληλη υλοποίηση στο τετραπύρηνο σύστημα ή/και έως 3,5 φορές μεγαλύτερη στο οκταπύρηνο σύστημα. Αντίστοιχα καλές επιδόσεις παρατηρήθηκαν για όλα τα σετ δεδομένων με εξαίρεση το \textit{web-BerkStan} graph όπου ο edge-coloring αλγόριθμος δημιούργησε ιδιαίτερα αυξημένο αριθμό ομάδων.

Σε όλες τις περιπτώσεις, η παράλληλη υλοποίηση του αλγορίθμου «στοχαστικοποίησης» των στηλών του γράφου για την δημιουργία του πίνακα $\bm{A}_s$ ήταν σαφώς ταχύτερη από τη σειριακή.


\section*{Επαλήθευση ορθότητας αποτελεσμάτων}
Ο σειριακός αλγόριθμος που υλοποιήθηκε υποστηρίζεται μαθηματικά σε κάθε του βήμα πράγμα που αρκεί για να θεωρηθεί σωστός. Σε κάθε περίπτωση, για να αποκλειστεί η πιθανότητα προγραμματιστικού λάθους, έγινε επαλήθευση αποτελεσμάτων για μικρής κλίμακας web-Graphs με προϋπάρχουσες υλοποιήσεις που βρέθηκαν στο διαδίκτυο σε γλώσσα προγραμματισμού \textit{MATLAB}. Αυτές οι υλοποιήσεις συμπεριλαμβάνονται στα παραδοτέα, καθώς και τα script που χρησιμοποιήθηκαν για τη σύγκριση των αποτελεσμάτων.

Η ορθότητα του παράλληλου αλγορίθμου μπορεί να επιβεβαιωθεί από την ταύτιση των αποτελεσμάτων της με αυτά του σειριακού. Συγκεκριμένα, έχουν πάντα ακριβώς ίδιο αριθμό από iterations, ίδια ακολουθία σύγκλισης και πρακτικά ίδιο αποτέλεσμα. Όλα τα παραπάνω μπορούν να επιβεβαιωθούν χρησιμοποιώντας το ειδικά δημιουργημένο \textit{python script} που περιλαμβάνεται στα παραδοτέα. Αυτό δημιουργεί γραφικές παραστάσεις από τα δεδομένα καταγραφής και τα αποτελέσματα των προγραμμάτων σαν αυτές που παρουσιάστηκαν στην παρούσα αναφορά.

Τέλος, έχει επιβεβαιωθεί (με το εργαλείο valgrind) ότι τα προγράμματα αυτά δεν παρουσιάζουν σφάλματα μνήμης.

\section*{Tαχύτητα σύγκλισης κι εκτέλεσης}
Η ταχύτητα σύγκλισης της μεθόδου gauss-seidel όπως φαίνεται στα αποτελέσματα είναι εντυπωσιακή. Το διάνυσμα της pagerank άρχισε να συγκλίνει από τις πρώτες κι όλας επαναλήψεις, με το τετράγωνο του μέτρου της διαφοράς των τελευταίων προσεγγίσεων να είναι μικρότερο -σε όλες τις περιπτώσεις- του $10^{-5}$ πριν την 7η επανάληψη. 

Αυτός ο   ρυθμός σύγκλισης την καθιστά ταχύτερη τόσο από, την κλασσική για το πρόβλημα, power method\parencite{silvestrepagerank} όσο και από την μέθοδο Jacobi που χρησιμοποιεί μόνο τις προηγούμενες τιμές κάθε $x_i$ σε κάθε επανάληψη.

Η επιτάχυνση της παράλληλης υλοποίησης, λαμβάνοντας υπόψη τις συνθήκες του αλγορίθμου, τον αριθμό των διαφορετικών χρωμάτων και τις επιταχύνσεις που επιτεύχθηκαν στη βιβλιογραφία\parencite{hasenplaugh2016parallel}, κρίνεται κι αυτή από ικανοποιητική έως πάρα πολύ καλή για τα τρία από τα τέσσερα web-graphs στα οποία εξετάστηκε.

\section*{Παραδοτέα}
Το repository με όλα τα αρχεία της εργασίας βρίσκεται
\href{https://github.com/nicktheway/Pagerank}{εδώ}.

Επιπλέον, το documentation του κώδικα σε C της εργασίας βρίσκεται
\href{https://nicktheway.github.io/Pagerank/html/index.html}{εδώ}.

\printbibliography

 
\end{document}
\clearpage
\appendix
\renewcommand\thefigure{A\arabic{figure}}
\setcounter{figure}{0}


\section*{Appendix}

Appendix content


\end{document}
