%\title{Two column article template by M.A.}

\documentclass[11pt, twocolumn]{article}

\usepackage{authblk}

\makeatletter
\renewcommand{\maketitle}{\bgroup\setlength{\parindent}{0pt}
\begin{flushleft}
  \textbf{\@title}

  \@author
\end{flushleft}\egroup
}
\makeatother

\title{\Huge Αλγόριθμος Pagerank με παράλληλη υλοποίηση της μεθόδου Gauss-Seidel \vspace{0.5cm}}
\author{\Large Νικόλαος Κατωμέρης \hspace{2cm} ΑΕΜ: 8551 \hspace{0.5cm} ngkatomer@auth.gr\normalsize \\ Παράλληλα και διανεμημένα συστήματα. \\Τμήμα ηλεκτρολόγων μηχανικών και μηχανικών υπολογιστών, ΑΠΘ}


\date{\today}

\usepackage[showframe=false, margin=.75in]{geometry}

\usepackage{polyglossia}
\newfontfamily\greekfontsf[Script=Greek]{texgyreheros-regular.otf}
\newfontfamily\greekfonttt[Script=Greek]{Latin Modern Mono}
\setdefaultlanguage{greek}
\setotherlanguage{english}
\setmainfont{Noto Serif}

\usepackage{amsmath}
\usepackage{bm}
\usepackage{abstract}
\renewcommand{\abstractname}{}    % clear the title
\renewcommand{\absnamepos}{empty} % originally center

\usepackage{lipsum}

\usepackage{afterpage}

\renewenvironment{abstract}
 {\small
  \begin{center}
  \bfseries \abstractname\vspace{-.5em}\vspace{0pt}
  \end{center}
  \list{}{%
    \setlength{\leftmargin}{10mm}% <---------- Change margin here
    \setlength{\rightmargin}{\leftmargin}%
  }%
  \item\relax}
 {\endlist}

\usepackage[utf8]{inputenc}
%\usepackage{graphicx}
\usepackage{wrapfig}
\usepackage[lofdepth,lotdepth]{subfig}
\usepackage{booktabs}

\usepackage{titlesec}
\titlespacing\section{0pt}{10pt plus 4pt minus 2pt}{0pt plus 2pt minus 2pt}

\usepackage{rotating}

\usepackage{caption}
%\usepackage{subcaption}

%\usepackage[ngerman]{babel}

\usepackage{csquotes}

\usepackage{todonotes}

\usepackage{stfloats}

%\usepackage[T1]{fontenc}
\usepackage{textcomp}
\usepackage{times}

\usepackage{framed} %um Boxen zu machen

\usepackage[backend=biber, style=ieee]{biblatex}
\addbibresource{references.bib}

%\renewbibmacro*{volume+number+eid}{%
%\printfield{volume}%
%  \setunit*{\adddot}% DELETED
%\setunit*{\addnbspace}% NEW (optional); there's also \addnbthinspace
%\printfield{number}%
%\setunit{\addcomma\space}%
%\printfield{eid}}
%\DeclareFieldFormat[article]{number}{\mkbibparens{#1}}


\usepackage{setspace}
\onehalfspacing

\setlength{\columnsep}{0.8cm}

\setlength{\parskip}{0em}

\usepackage{color}
\definecolor{black}{gray}{0} % 10% gray

\usepackage[colorlinks=true,linkcolor=black,citecolor=black]{hyperref}

\usepackage{tabularx}
\newcolumntype{b}{X}
\newcolumntype{s}{>{\hsize=.5\hsize}X}

\usepackage{ntheorem}
\newtheorem*{TRQ}{Research Question}

\newtheorem{Hyp}{Hypothesis} 


\begin{document}

\twocolumn[
\begin{@twocolumnfalse}
\maketitle
\begin{abstract}
%\textit{\lipsum[2]
%\bigskip}
Στην παρούσα αναφορά περιγράφεται η διαδικασία παραλληλοποίησης του αλγορίθμου Gauss-Seidel για τον υπολογισμό του διανύσματος Pagerank γράφων του διαδικτύου. Επίσης, παρουσιάζονται αποτελέσματα σε ενδεικτικούς γράφους καθώς και αποδείξεις ορθότητας του υλοποιημένου προγράμματος.
\end{abstract}
\end{@twocolumnfalse}
]

\section*{Pagerank}
Η βασική ιδέα του αλγορίθμου Pagerank είναι η αξιολόγηση της κάθε σελίδας του διαδικτύου με βάση τον αριθμό των συνδέσμων προς αυτήν και την ποιότητα αυτών, η οποία καθορίζεται από την αξιολόγηση των σελίδων που τους περιέχουν.

Το αποτέλεσμα της παραπάνω αξιολόγησης ταυτίζεται με το ποσοστό του χρόνου στον οποίο ένας τυχαίος περιπατητής θα βρισκόταν στην εκάστοτε σελίδα, αν ξεκινώντας από κάποια τυχαία σελίδα, ακολουθούσε τυχαία κάποιον από τους συνδέσμους της, μετά το ίδιο για την επόμενη κι ούτω καθ' εξής.

Η παραπάνω διαδικασία μπορεί να προσομοιωθεί με μία αλυσίδα Markov της οποίας ο πίνακας μεταβάσεων αποτελείται από στοιχεία που δείχνουν την πιθανότητα μετάβασης από μία σελίδα σε μία άλλη.

Έστω $\bm{A}$, ο $N\times N$  πίνακας μεταβάσεων της αλυσίδας για $N$ σελίδες του διαδικτύου. Αν  $ j\rightarrow i$ σημαίνει ότι η σελίδα $j$ περιέχει σύνδεσμο προς τη σελίδα $i$, ο $\bm{A}$ ορίζεται αρχικά ως εξής:

\[
  \bm{A}(i, j) = 
  \begin{cases}
    1, & \text{αν } j\rightarrow i \\
    0, & \text{αλλιώς }
  \end{cases}
\]

Επειδή όμως ο παραπάνω πίνακας θέλουμε να εκφράζει την πιθανότητα ο περιπατητής να πάει από τη σελίδα $j$ στη σελίδα $i$ σε ένα βήμα, είναι απαραίτητο το άθροισμα των στοιχείων κάθε στήλης του πίνακα να ισούται με $1$.

Οπότε, αφού λάβουμε τον πίνακα με τους συνδέσμους των σελίδων, διαιρούμε όλα του τα στοιχεία με το άθροισμα των στοιχείων της εκάστοτε στήλης $j$, έστω $L_j$, και παίρνουμε τον «στοχαστικοποιημένο» πίνακα $\bm{A}_s$:

\[
  \bm{A}_s(i, j) = 
  \begin{cases}
    \frac{1}{L_j}, & \text{αν } j\rightarrow i \\
    0, & \text{αλλιώς }
  \end{cases}
\]

Επιπλέον, καθώς ο περιπατητής ξεκινά από μία σελίδα, είναι πολύ πιθανό, ακολουθώντας συνεχώς συνδέσμους να είναι αδύνατο να μεταβεί σε όλες τις σελίδες του γράφου. Το γεγονός αυτό, συν το γεγονός ότι ένας χρήστης του διαδικτύου δεν ακολουθεί μόνο συνδέσμους, αλλά «τηλεμεταφέρεται» κιόλας σε άλλες σελίδες, καθιστά τον παραπάνω πίνακα ανεπαρκή για την προσομοίωση της τυχαίας διαδρομής ενός χρήστη του διαδικτύου.

Έτσι, εισάγεται στον αλγόριθμο μια σταθερά «τηλεμεταφοράς» που δηλώνει την πιθανότητα ο περιπατητής να μεταβεί σε κάποια τυχαία σελίδα του διαδικτύου χωρίς να υπάρχει κάποιος σύνδεσμος προς αυτήν στη σελίδα που βρίσκεται.

Η σταθερά αυτή, έστω $c$, μπορεί να επιλεγεί ως όρισμα στο πρόγραμμα υπολογισμού του διανύσματος pagerank που υλοποιήθηκε. Στην περίπτωση που δεν δοθεί, θα έχει την τιμή 0.15 που έχει δειχθεί ότι αποτελεί καλή επιλογή \parencite{brin1998anatomy}.

Με την εισαγωγή της παραπάνω σταθεράς, ο πίνακας μεταβάσεων πλέον θα είναι:
\[
  \bm{A}_{final} = \frac{c}{N}\cdot\bm{1}_{N\times N}+(1-c)\bm{A}_s
\]

Στην παραπάνω σχέση, η σταθερά $c$ διαιρείται με τον αριθμό των γραμμών $N$ έτσι ώστε οι στήλες και του τελικού πίνακα $\bm{A}_{final}$ να έχουν στοιχεία με άθροισμα $1$.

Με τη χρήση του παραπάνω πίνακα είναι δυνατό να υπολογιστεί ένα διάνυσμα $N\times1$ πολλαπλασιάζοντάς τον πίνακα με ένα αρχικό διάνυσμα -πχ ένα ομοιόμορφο- αρκετές φορές, έως ότου το αποτέλεσμα συγκλίνει. Το διάνυσμα αυτό αποτελεί ιδιοδιάνυσμα του πίνακα $\bm{A}$ για την ιδιοτιμή 1 και αποτελεί το ζητούμενο διάνυσμα με την αξιολόγηση της κάθε σελίδας (κόμβου).

Για να συγκλίνει με βεβαιότητα το διάνυσμα απαιτείται ακόμη ένα βήμα, θα πρέπει, αν ο περιπατητής βρεθεί σε κάποια σελίδα που δεν περιέχει καμία σύνδεση, να τηλεμεταφερθεί σε μία τυχαία σελίδα με βάση ένα διάνυσμα πιθανοτήτων. Στην υλοποίηση αυτή, ο περιπατητής έχει την ίδια πιθανότητα να τηλεμεταφερθεί σε οποιαδήποτε σελίδα.

Με όλα τα παραπάνω δεδομένα, αποδεικνύεται ότι ο τελικός πίνακας θα έχει μία ιδιοτιμή ένα και το πρόβλημα πλέον έγκειται στην εύρεση του ιδιοδιανύσματός της.

Για την εύρεση του ιδιοδιανύσματος, όπως ζητείται, χρησιμοποιείται η μέθοδος Gauss-Seidel.



\section*{Gauss Seidel}
H μέθοδος Gauss-Seidel είναι μια αναδρομική μέθοδος επίλυσης γραμμικών συστημάτων. Ο αναδρομικός αλγόριθμός της μεθόδου για την επίλυση ενός γραμμικού συστήματος της μορφής:
$$\bm{A}\vec{x}=\vec{b}$$
είναι για το $i$-στο στοιχείο στο $k+1$ βήμα:
$$x_i(k+1) = \frac{1}{A_{ii}}\bigg(b_i-\sum_{j=1}^{i-1}A_{ij}x_j(k+1)-\sum_{j=i+1}^{N}A_{ij}x_j(k)\bigg)$$

Όπως φαίνεται, ο αλγόριθμος αυτός μοιάζει καθαρά σειριακός αφού ο υπολογισμός του κάθε στοιχείου σε κάθε βήμα εξαρτάται από τον υπολογισμό των νέων τιμών των στοιχείων πριν απ' αυτό.
Για περισσότερες λεπτομέρειες σχετικά με τη μέθοδο αυτή και τα κριτήρια σύγκλισής της ανατρέξτε στον \textcite{saad2003iterative}.

Στην περίπτωσή μας, για την εύρεση της pagerank με τη μέθοδο Gauss-Seidel, θα πρέπει πρώτα να το εκφράσουμε σαν σύστημα γραμμικών εξισώσεων. Όπως προαναφέρθηκε το διάνυσμα Pagerank αποτελεί ιδιοδιάνυσμα του πίνακα $\bm{A}$ για την ιδιοτιμή 1. Επομένως, ισχύει ότι:\\
\begin{multline*}
\bm{A}_{final}\vec{x} = \vec{x}\Rightarrow (\bm{I}-\bm{A}_{final})\vec{x} = 0 
\Rightarrow \\
\bigg(\bm{I}-\frac{c}{N}\cdot\bm{1}_{N\times N}-(1-c)\bm{A}_n\bigg)\vec{x} = 0
\end{multline*}

Όμως, το $\bm{1}_{N\times N}\cdot\vec{x}$ είναι πάντα ίσο με $\bm{1}_{N\times1}$ επειδή το άθροισμα των στοιχείων του $\vec{x}$, ως άθροισμα όλων των πιθανοτήτων, εκφράζει την πιθανότητα ο περιπατητής να βρίσκεται σε μια οποιαδήποτε σελίδα.
 Έτσι, το σύστημά μας παίρνει τη μορφή:

\begin{equation} \label{eq:1}
(\bm{I}-(1-c)\bm{A}_n)\vec{x} = \frac{c}{N}\bm{1}_{N\times1}
\end{equation}

που μπορεί να λυθεί αναδρομικά με τη μέθοδο Gauss-Seidel, αρκεί διασφαλιστεί ότι σε κάθε βήμα $\bm{1}_{N\times N}\cdot\vec{x}_k =\bm{1}_{N\times1}$, δηλαδή ότι το άθροισμα των στοιχείων του $\vec{x}_k$ παραμένει ένα.


\begin{Hyp}
A hypothesis is likely to be false if it is longer than this line
\label{hyp:nd_conflictivity}
\end{Hyp}

\lipsum[1]

\begin{table}[ht]
\centering
\begin{tabular}{rrrrrr}
  \toprule
  text & text & text & text & text & text  \\
  \midrule
  1 &  84 &  46 &   4 &   3 &   5 \\
    2 &  67 &  24 &   5 &   1 &   5 \\
    3 &  54 &  26 &   4 &   2 &   5 \\
    4 &  44 &  16 &   5 &   2 &   5 \\
    5 &  53 &  21 &   6 &   2 &   5 \\
   \bottomrule
\end{tabular}
\caption{An example single column table}
\label{tab:example}
\end{table}

\lipsum[1]

\begin{figure*}[ht!]
\centering
\includegraphics[width=1\textwidth]{plots/google_times.png}
\caption{A two column figure (indicate with figure*)}
\label{fig:example}
\end{figure*}

\lipsum[1]

\printbibliography


 
\end{document}
\clearpage
\appendix
\renewcommand\thefigure{A\arabic{figure}}
\setcounter{figure}{0}


\section*{Appendix}

Appendix content


\end{document}
