Για την υλοποίηση του προγράμματος υπολογισμού της pagerank ενός συνόλου ιστοσελίδων (κόμβων) θα πρέπει τα δεδομένα των συνδέσμων μεταξύ των κόμβων, δηλαδή ο πίνακας $\bm{A}$, να αποθηκευτούν με κάποιο τρόπο στη μνήμη.

Δεδομένου ότι ο πίνακας αυτός είναι πολύ αραιός, αλλά και λόγω του γεγονότος ότι η μέθοδος gauss-seidel απαιτεί, σε κάθε της βήμα και για κάθε κόμβο, πολλαπλασιασμούς στοιχείων μίας γραμμής του $\bm{A}$, επιλέχθηκε η χρήση της δομής «Compressed Row Storage (CRS)». 

Η δομή αυτή επιτρέπει την αποθήκευση πινάκων απαιτώντας χώρο στη μνήμη ανάλογο με τα μη μηδενικά στοιχεία αυτών. Επιπλέον, διατρέχοντας τον πίνακα ανά γραμμή, όλα τα στοιχεία της γραμμής βρίσκονται συνεχόμενα στη μνήμη καθιστώντας τη δομή αυτή φιλική προς την μνήμη cache των επεξεργαστών.