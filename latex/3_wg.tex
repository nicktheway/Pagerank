Το πρόβλημα με την \fref{eq:1} είναι ότι ο πίνακας $(\bm{I}-(1-c)\bm{A}_n)$ δεν είναι αραιός καθώς ο $\bm{A}_n$ , όπως αναφέρθηκε νωρίτερα, έχει γεμάτες τις στήλες με τις σελίδες που στον $\bm{A}_s$ δεν είχαν καμία σύνδεση. Επίσης, το σύνολο αυτών των σελίδων στο διαδίκτυο είναι πολύ μεγάλο\parencite{eiron2004ranking}, με αποτέλεσμα ο $\bm{A}_n$ να απαιτεί πολύ περισσότερο χώρο στη μνήμη από τον $\bm{A}_s$.

Έχουν προταθεί διάφορες λύσεις για την αποφυγή αυτού του φαινομένου, όπως για παράδειγμα να προστεθεί σ' όλες αυτές τις σελίδες σύνδεσμος προς τον εαυτό τους. Μια άλλη πρόταση, η οποία και προτιμήθηκε, χρησιμοποιεί μόνο τον πίνακα $\bm{A}_s$ ο οποίος είναι αραιός και καταλήγει στην ίδια ακριβώς λύση με την \fref{eq:1}. Όπως αποδεικνύεται \parencite{del2005fast}, η λύση της \fref{eq:1} ισοδυναμεί με $\frac{\vec{y}}{\Vert\vec{y}\Vert}_1$, όπου $\vec{y}$ η λύση της:
\begin{equation}\label{eq:2}
(\bm{I}-(1-c)\bm{A}_s)\vec{y} = \frac{1}{N}\cdot\bm{1}_{N\times1}
\end{equation}

H οποία απαιτεί την αποθήκευση μόνο του πίνακα $\bm{A}_s$. Δεδομένου ότι ο πίνακας αυτός είναι πολύ αραιός, αλλά και λόγω του γεγονότος ότι η μέθοδος gauss-seidel έχει, σε κάθε της βήμα και για κάθε κόμβο, πολλαπλασιασμούς στοιχείων μίας γραμμής του $\bm{A}_s$, επιλέχθηκε η χρήση της δομής «Compressed Row Storage (CRS)». 

Η δομή αυτή επιτρέπει την αποθήκευση πινάκων σε χώρο στη μνήμη ανάλογο με τα μη μηδενικά στοιχεία αυτών. Επιπλέον, διατρέχοντας τον πίνακα ανά γραμμή, όλα τα στοιχεία της γραμμής βρίσκονται συνεχόμενα στη μνήμη καθιστώντας τη δομή αυτή φιλική προς την μνήμη cache των επεξεργαστών.