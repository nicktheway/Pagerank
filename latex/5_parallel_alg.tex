Η μέθοδος gauss-seidel γενικώς είναι μια σειριακή μέθοδος. Σε περιπτώσεις όμως που ο πίνακας $\bm{A}$ είναι αραιός, όπως και στην προκειμένη, είναι δυνατός ο παράλληλος υπολογισμός των $x_i$ που δεν αλληλοεξαρτώνται και εξαρτώνται μόνο από στοιχεία $x_j\text{με } j < i$ των οποίων η τιμή έχει ήδη ενημερωθεί ή δεν χρειάζεται να ενημερωθεί ($j > i$).

Για την εύρεση των ομάδων αυτών με τα μη αλληλοεξαρτώμενα στοιχεία του $x$ υλοποιήθηκε ο εξής απλός άπληστος αλγόριθμος:

\begin{itemize}
\item Διατρέχουμε κατά αύξουσα σειρά τα στοιχεία (κόμβους) του γράφου.
\item Τον 1ο κόμβο τον εισάγουμε στην 1η ομάδα.
\item Tον 2ο κόμβο τον εισάγουμε κι αυτόν στην 1η ομάδα εκτός αν υπάρχει σύνδεσμος μεταξύ αυτού και του 1ου. Σε εκείνη την περίπτωση τον τοποθετούμε στην 2η ομάδα.
\item Συνεχίζουμε με τους υπόλοιπους κόμβους, τοποθετώντας τους κάθε φορά στην Λ+1 ομάδα όπου Λ η μεγαλύτερη ομάδα των κόμβων που συνδέονται μ' αυτούς.
\end{itemize}

Αφού, γίνει ο ο «χρωματισμός» των κόμβων σε ομάδες προχωράμε σε ανακατάταξη των κόμβων έτσι ώστε τα στοιχεία της κάθε ομάδας να βρίσκονται συνεχόμενα στη μνήμη και με αύξουσα σειρά ομάδας.

Ύστερα, η εύρεση του διανύσματος pagerank είναι όμοια με τον σειριακό αλγόριθμο με τη διαφορά ότι εδώ υπολογίζουμε παράλληλα τις τιμές $\vec{x}_i(k+1)$ σε κάθε βήμα για τα $i$ που βρίσκονται στην ίδια ομάδα. Επίσης, παράλληλα γίνονται και όλες οι πράξεις πινάκων που απαιτούνται για τον έλεγχο σύγκλισης.

Τέλος, οι κόμβοι επιστρέφουν στην αρχική τους αλληλουχία ώστε να γίνει εύκολα σύγκριση του αποτελέσματος με τον σειριακό αλγόριθμο.
