Αρχικά, ο αλγόριθμος υπολογισμού της pagerank που υλοποιήθηκε υποστηρίζεται μαθηματικά σε κάθε του βήμα. Σε κάθε περίπτωση, για να αποκλειστεί η πιθανότητα προγραμματιστικού λάθους, έγινε επαλήθευση αποτελεσμάτων για μικρής κλίμακας web-Graphs με προϋπάρχουσες υλοποιήσεις \parencite{mastersthesis} σε γλώσσα προγραμματισμού \textit{MATLAB}. Αυτές οι υλοποιήσεις συμπεριλαμβάνονται στα παραδοτέα, καθώς και τα script που χρησιμοποιήθηκαν για τη σύγκριση των αποτελεσμάτων.

Η ορθότητα του παράλληλου αλγορίθμου μπορεί να επιβεβαιωθεί κι από την ταύτιση των αποτελεσμάτων του με αυτά του σειριακού, όπως φαίνεται στο \fref{fig:dif}. Επιπλέον, έχουν πάντα ακριβώς ίδιο αριθμό από iterations και ίδια ακολουθία σύγκλισης όπως φαίνεται στα \fref{fig:relat} και \fref{fig:relatd}.

Καθώς ο σκοπός της εργασίας ήταν η παραλληλοποίηση του αλγορίθμου, δεν έγινε αλλαγή στην αλληλουχία των κόμβων με τρόπο που θα διαφοροποιούσε το παράλληλο από το σειριακό πρόγραμμα, καθώς η σύγκλιση της μεθόδου gauss-seidel εξαρτάται από αυτή την αλληλουχία. Έτσι, μπορεί να επιβεβαιωθεί η ταύτιση των αποτελεσμάτων των δύο υλοποιήσεων.

Επιπλέον, μπορεί να επιβεβαιωθεί και πειραματικά η απουσία race condition στην παράλληλη υλοποίηση από το γεγονός ότι συγκλίνει πάντα με τον ίδιο τρόπο που ταυτίζεται και μ' αυτόν της σειριακής.

Τέλος, έχει επιβεβαιωθεί (με το εργαλείο valgrind) ότι τα προγράμματα αυτά δεν παρουσιάζουν σφάλματα μνήμης.
