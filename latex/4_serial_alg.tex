Με βάση τα παραπάνω, υλοποιήθηκε το σειριακό πρόγραμμα υπολογισμού της pagerank. Συνοψίζεται ως εξής:
\begin{enumerate}

\item Επιλογή γράφου δεδομένων εισόδου (σύνδεσμοι) και προαιρετικά συγκεκριμένου αριθμού επαναλήψεων ή σταθερών τηλεμεταφοράς ($c$) και σύγκλισης ($E$). Οι προεπιλεγμένες τιμές είναι: \\ $c_{def} = 0.15, E_{def} = 1e-12$

\item Φόρτωση στη μνήμη των δεδομένων εισόδου σε δομή CRS. Tα δεδομένα αυτά αποτελούν τον αρχικό πίνακα $\bm{A}$.

\item Μετατροπή του $\bm{A}$ στον $\bm{A}_s$ και πολλαπλασιασμός του με την σταθερά $-(1-c)$. Έστω $a_{ij}$ τα στοιχεία του πίνακα που προκύπτει.

\item Δημιουργία των $\vec{b},\vec{x}_0=\frac{1}{N}\bm{1}_{N\times1}$

\item Εκτέλεση αναδρομικού αλγορίθμου της μεθόδου Gauss-Seidel έως ότου $\Vert\vec{x}_{k+1}-\vec{x}_k\Vert^2 < E$ ή $k > k_{max}$, όπου έχει προκαθοριστεί $k_{max} = 150$ για ασφάλεια. Επίσης, αν έχει δοθεί συγκεκριμένος αριθμός επαναλήψεων, θα χρησιμοποιηθεί αυτός.

Σε κάθε βήμα του αλγορίθμου:
\begin{itemize}[leftmargin=*]
\item Yπολογίζεται το $x_{k+1}$:

\hfil$\displaystyle x_i(k+1) = \frac{1}{1-a_{ii}}\bigg(b_i-\sum_{j=1}^{i-1}a_{ij}x_j(k+1)-\sum_{j=i+1}^{N}a_{ij}x_j(k)\bigg)$\hfil

\item Κάθε στοιχείο $x_{k+1}$ διαιρείται με το άθροισμα όλων των στοιχείων του, ώστε το νέο $x_{k+1}$ να ισχύει η προϋπόθεση ότι το άθροισμα των στοιχείων του $x$ σε κάθε βήμα πρέπει να ισούται με 1.

\item Υπολογίζεται το $\Vert\vec{x}_{k+1}-\vec{x}_k\Vert^2$ και αυξάνεται το $k$ κατά 1.

\end{itemize}
\end{enumerate}