Η βασική ιδέα του αλγορίθμου Pagerank είναι η αξιολόγηση της κάθε σελίδας του διαδικτύου με βάση τον αριθμό των συνδέσμων προς αυτήν και την ποιότητα αυτών, η οποία καθορίζεται από την αξιολόγηση των σελίδων που τους περιέχουν.

Το αποτέλεσμα της παραπάνω αξιολόγησης ταυτίζεται με το ποσοστό του χρόνου στον οποίο ένας τυχαίος περιπατητής θα βρισκόταν στην εκάστοτε σελίδα, αν ξεκινώντας από κάποια τυχαία σελίδα, ακολουθούσε τυχαία κάποιον από τους συνδέσμους της, μετά το ίδιο για την επόμενη κι ούτω καθ' εξής.

Η παραπάνω διαδικασία μπορεί να προσομοιωθεί με μία αλυσίδα Markov της οποίας ο πίνακας μεταβάσεων αποτελείται από στοιχεία που δείχνουν την πιθανότητα μετάβασης από μία σελίδα σε μία άλλη.

Έστω $\bm{A}$, ο $N\times N$  πίνακας μεταβάσεων της αλυσίδας για $N$ σελίδες του διαδικτύου. Αν  $ j\rightarrow i$ σημαίνει ότι η σελίδα $j$ περιέχει σύνδεσμο προς τη σελίδα $i$, ο $\bm{A}$ ορίζεται αρχικά ως εξής:

\[
  \bm{A}(i, j) = 
  \begin{cases}
    1, & \text{αν } j\rightarrow i \\
    0, & \text{αλλιώς }
  \end{cases}
\]

Επειδή όμως ο παραπάνω πίνακας θέλουμε να εκφράζει την πιθανότητα ο περιπατητής να πάει από τη σελίδα $j$ στη σελίδα $i$ σε ένα βήμα, είναι απαραίτητο το άθροισμα των στοιχείων κάθε στήλης του πίνακα να ισούται με $1$.

Οπότε, αφού λάβουμε τον πίνακα με τους συνδέσμους των σελίδων, διαιρούμε όλα του τα στοιχεία με το άθροισμα των στοιχείων της εκάστοτε στήλης $j$, έστω $L_j$, και παίρνουμε τον «στοχαστικοποιημένο» πίνακα $\bm{A}_s$:

\[
  \bm{A}_s(i, j) = 
  \begin{cases}
    \frac{1}{L_j}, & \text{αν } j\rightarrow i \\
    0, & \text{αλλιώς }
  \end{cases}
\]

Επιπλέον, καθώς ο περιπατητής ξεκινά από μία σελίδα, είναι πολύ πιθανό, ακολουθώντας συνεχώς συνδέσμους να είναι αδύνατο να μεταβεί σε όλες τις σελίδες του γράφου. Το γεγονός αυτό, συν το γεγονός ότι ένας χρήστης του διαδικτύου δεν ακολουθεί μόνο συνδέσμους, αλλά «τηλεμεταφέρεται» κιόλας σε άλλες σελίδες, καθιστά τον παραπάνω πίνακα ανεπαρκή για την προσομοίωση της τυχαίας διαδρομής ενός χρήστη του διαδικτύου.

Έτσι, εισάγεται στον αλγόριθμο μια σταθερά «τηλεμεταφοράς» που δηλώνει την πιθανότητα ο περιπατητής να μεταβεί σε κάποια τυχαία σελίδα του διαδικτύου χωρίς να υπάρχει κάποιος σύνδεσμος προς αυτήν στη σελίδα που βρίσκεται.

Η σταθερά αυτή, έστω $c$, μπορεί να επιλεγεί ως όρισμα στο πρόγραμμα υπολογισμού του διανύσματος pagerank που υλοποιήθηκε. Στην περίπτωση που δεν δοθεί, θα έχει την τιμή 0.15 που έχει δειχθεί ότι αποτελεί καλή επιλογή \parencite{brin1998anatomy}.

Με την εισαγωγή της παραπάνω σταθεράς, ο πίνακας μεταβάσεων πλέον θα είναι:
\[
  \bm{A}_{final} = \frac{c}{N}\cdot\bm{1}_{N\times N}+(1-c)\bm{A}_n
\]

Στην παραπάνω σχέση, η σταθερά $c$ διαιρείται με τον αριθμό των γραμμών $N$ έτσι ώστε οι στήλες και του τελικού πίνακα $\bm{A}_{final}$ να έχουν στοιχεία με άθροισμα $1$.

Με τη χρήση του παραπάνω πίνακα είναι δυνατό να υπολογιστεί ένα διάνυσμα $N\times1$ πολλαπλασιάζοντάς τον πίνακα με ένα αρχικό διάνυσμα -πχ ένα ομοιόμορφο- αρκετές φορές, έως ότου το αποτέλεσμα συγκλίνει. Το διάνυσμα αυτό αποτελεί ιδιοδιάνυσμα του πίνακα $\bm{A}$ για την ιδιοτιμή 1 και αποτελεί το ζητούμενο διάνυσμα με την αξιολόγηση της κάθε σελίδας (κόμβου).

Τα παραπάνω δεν θα ισχύουν αν ο πίνακας $\bm{A}_s$ περιέχει σελίδες χωρίς καμία σύνδεση. Σε αυτή την περίπτωση, επιθυμούμε ο περιπατητής να τηλεμεταφερθεί σε μία τυχαία σελίδα. Οπότε, προτού υπολογιστεί ο $\bm{A}_{final}$, είναι απαραίτητο οι μηδενικές στήλες του $\bm{A}_s$ να αντικατασταθούν με νέες στήλες που αποτελούνται από στοιχεία $\frac{1}{N}$. Έστω $\bm{A}_n$ ο πίνακας που δημιουργείται.

Με όλα τα παραπάνω δεδομένα, αποδεικνύεται ότι ο τελικός πίνακας θα έχει μοναδική ιδιοτιμή ένα και το πρόβλημα πλέον έγκειται στην εύρεση του ιδιοδιανύσματός της.

Για την εύρεση του ιδιοδιανύσματος, όπως ζητείται, χρησιμοποιείται η μέθοδος Gauss-Seidel.

